%% Template for an article following the Dino Documentation
%% Guidelines.
%%
%% Please read the Dino Documentation Guidelines first before writing
%% any Dino document as it is mandatory to fully make use of the dino.sty
%% document style. This especially applies for the macros in there
%% that define images, references, etc...

\documentclass[a4paper]{article} %% never change options or the
%% document class!!!

\usepackage[dinodraft]{dino}                %% never change this line!!!

%% insert necessary additional packages here - please keep in mind
%% that dino.sty itself already includes the necessary standard
%% packages like graphicx, etc... please read the Dino
%% Documentation Guidelines for details.

%% never change any latex variables like pagestyle, baselineskip,
%% etc... here!!!

\begin{document}
\begin{Form}\end{Form}
%% ======================================== start document header

\title{Software Design of the GPSTool Package}                     %% insert title here
\author{(\cdaller)\\                         %% insert names of authors here in
  %% Dino Document Style form,
  %% e.g. \cdaller, \hhaub
  $Revision$}                      %% never change this line!!!


\maketitle

\vspace{1cm}

\begin{abstract}

This document describes the rough design of the modules in the
\packagename{org.dinopolis.gpstool} package and the modules of the
\classname{GPSMap} application.

\end{abstract}

\newpage

\tableofcontents

\newpage

%######################################################################
%######################################################################
\section{Architectural Design}
\label{SoftwareDesignOfTheGpstoolPackage-ArchitecturalDesign}

This section describes the modules contained in the
\packagename{org.dinopolis.gpstool} package and explains the structure
of the \classname{GPSMap} application.

%######################################################################
\subsection{GPS Data Sources}
\label{SoftwareDesignOfTheGpstoolPackage-GpsDataSources}

One of the major modules in the \packagename{org.dinopolis.gpstool}
package is the one that reads and interprets data from a gps
device. This module is named
\packagename{org.dinopolis.gpstool.gpsinput}.

The module was designed to be independent of the format of the data
and of the source of the data. An example for different formats of the
data could be NMEA or the proprietary Garmin protocol, the source
could either be the serial port, a file or a network server that
provides any clients with gps data (like
\externalref{gpsd}{http://freshmeat.net/projects/gpsd/} does).

So to be able to get gps information (like position, altitude, speed,
etc.), the source in the form of a \classname{GPSDevice} has to be
chosen and a way to interpret the data coming from the device, in the
form of a \classname{GPSDataProcessor}.

These two classes are connected and from this moment on, gps
information can be obtained. This information is delivered in the form
of events, anyone can register for. The listener can register for all
gps events or just for a specific one. In
\stdref{tab:SoftwareDesignOfTheGpstoolPackage-EventsFiredFromTheGpsdataprocessor}
are the events and its value types listed.


\begin{table}[ht]
  \begin{center}
    \begin{tabular}{|l|l|}
      \hline
      Event Type & Value of Event \\
      \hline
      Location & \classname{GPSPosition}\\
      Heading & \classname{Float}\\
      Speed & \classname{Float}\\
      Number of Satellites & \classname{Integer}\\
      Altitude (in meters)& \classname{Float}\\
      Satellite Info & \classname{SatelliteInfo}\\
      Depth & \classname{Float}\\
      \hline
    \end{tabular}
  \end{center}
  \caption{Events fired from the GPSDataProcessor}
  \label{tab:SoftwareDesignOfTheGpstoolPackage-EventsFiredFromTheGpsdataprocessor}
\end{table}


A short code snipped shows how to read NMEA data from a serial device:

\begin{lstlisting}{Read gps data from the serial device.}
// create processor for NMEA data:
  GPSDataProcessor gps_data_processor = new GPSNmeaDataProcessor();

// create gps device for serial port:
  Hashtable environment = new Hashtable();
  environment.put(GPSSerialDevice.PORT_NAME_KEY,"/dev/ttyS1");
  environment.put(GPSSerialDevice.PORT_SPEED_KEY,new Integer(4800));
  GPSDevice gps_device = new GPSSerialDevice();
  gps_device.init(environment);

// connect processor with device and open it:
  gps_data_processor.setGPSDevice(gps_device);
  gps_data_processor.open();


// create property change listener for gps events:
PropertyChangeListener listener = new ProperyChangeListener()
{
  public void propertyChange(PropertyChangeEvent event)
  {
    Object value = event.getNewValue();
    String name = event.getPropertyName();
    if(name.equals(GPSDataProcessor.LOCATION))
    { 
      System.out.println("The new location is"
       +(GPSPosition)value.getLatitude() + "/"
       +(GPSPosition)value.getLongitude());
    }
  }
};

// register as listener for location events:
	gps_data_processor.addGPSDataChangeListener(GPSDataProcessor.LOCATION,listener);
\end{lstlisting}

A little example that demonstrates the features of this module is the
java application
\classname{org.dinopolis.gpstool.gpsinput.GPSTool}. It shows how to
read from a file or from the serial interface and how to register for
gps events. As a matter of fact, this application was the beginning of
the whole module.


%######################################################################
\subsection{GPSMap Application}
\label{SoftwareDesignOfTheGpstoolPackage-GpsmapApplication}

GPSMap is the main application of the \packagename{org.dinopolis.gpstool}
package. It is a moving map application that is able to show the
current position on maps that may be downloaded from the internet, a
track of the positions in the past, location markers for points of
interest, etc.

GPSMap uses some parts of the open source
\externalref{openmap}{http://openmap.bbn.com} framework. Although the
openmap framework provides a lot of functionality, some was not reused
but re-implemented to keep the dependencies to the library low.

Nevertheless, GPSMap uses openmap's \classname{MapBean} class as its
central component. A MapBean consists of layers that hold geographic
information to be drawn for a specific area and scale. 

The main class of the GPSMap application is
\classname{org.dinopolis.gpstool.GPSMap}.

%----------------------------------------------------------------------
%----------------------------------------------------------------------
\subsubsection{Resources}
\label{SoftwareDesignOfTheGpstoolPackage-Resources}

GPSMap reads some command line parameters, but most if the
configuration is read from a properties file
(\texttt{GPSMap.properties}). This file must be in the classpath of
the application and is read via the
\classname{org.dinopolis.util.Resources} class. Any changes of the
configuration are saved into a file into the directory
\texttt{.gpsmap} under the user' home directory. Not all resources can
be edited via the ``Preferences'' dialog, so if you are missing some
screws to turn, try the file itself.

The resources also hold the information for the resource editor
(title, description, type).


%----------------------------------------------------------------------
%----------------------------------------------------------------------
\subsubsection{User Interface}
\label{SoftwareDesignOfTheGpstoolPackage-UserInterface}

The user interface is widely configured in the resource files. The
structure of the menu is completely defined in the resource file and
the actions that are executed by selecting a menu entry are named in
the resource file as well.

Localization can be done by creating a localized version of the
resource file.

%----------------------------------------------------------------------
%----------------------------------------------------------------------
\subsubsection{Projection}
\label{SoftwareDesignOfTheGpstoolPackage-Projection}

This data is projected from the geoid coordinates (latitude,
longitude) to screen coordinates. As the projections provided by
openmap did not work for the maps of
\externalref{mapblast}{http://www.mapblast.com} or
\externalref{expedia}{http://www.expedia.com}, a new projection was
developed. The maths was taken from the
\externalref{gpsdrive}{http://www.gpsdrive.de} project of Fritz
Ganter.

This projection provides the calculation from latitude/longitude to
screen (\methodname{forward} methods) and from screen coordinates to
latitude/longitude (\methodname{inverse} methods).

The class that implements the projection is
\classname{org.dinopolis.gpstool.projection.FlatProjection}. 

For a full understanding of this class it is necessary to read the
documentation of the projections of the openmap framework.

%----------------------------------------------------------------------
%----------------------------------------------------------------------
\subsubsection{Layers}
\label{SoftwareDesignOfTheGpstoolPackage-Layers}

GPSMap organizes its data in layers that are administered by a
\classname{com.bbn.openmap.MapBean}. Whenever the projection changes
(scale or center is changed), the map bean informs all layers about
this change (\methodname{projectionChanged} method). The layers have
to recalculate (project) their data from latitude/longitude to the
screen coordinates and paint them. As the calculation may take its
time, this is usually done in a different task by a
\classname{SwingWorker}. As soon as the calculation is done, the data
is painted on the screen (\methodname{paintComponent} method).

The usage of background tasks also explains the behavior of GPSMap,
that after panning the map, other elements (in other layers) are drawn
slightly later at their correct position.

If one wants to add geographic information (e.g.~position of
friends/cars, etc.) the best solution is to add a new layer that
implements the \methodname{projectionChanged} and the
\methodname{paintComponent} methods. That's all! Using the projection
passed in the \methodname{projectionChanged} method, the conversion of
geographical to screen coordinates is easy. Lengthy calculations
should use a \classname{SwingWorker}, so the user interface is not
blocked. 

In the following, some detailed information about different layers is
given.

%----------------------------------------------------------------------
%----------------------------------------------------------------------
\subsubsection{Map Layer}
\label{SoftwareDesignOfTheGpstoolPackage-Map-Layer}

The map layer is probably the most important layer at the moment. It
displays raster maps that were previously downloaded form expedia or
mapblast and stored locally on the hard disk (directory
\path{<home>/.gpsmap/maps}). The informations about the files is kept
in the file \path{<home>/.gpsmap/maps.txt} (name of file,
latitude/longitude of center of map, scale of map (in mapblast style),
height/width of image). In this file, relative and absolute paths are
accepted for maps.

One principle of the map painting algorithm is that if no maps for a
given scale are available, maps of other scales are used as well and
resized to fit the used scale (see
\stdref{fig:SoftwareDesignOfTheGpstoolPackage-MapsOfDifferentScalesMayBeDisplayed}
for an example).

\image[1.0]{images/screenshot_diff_scales}{Maps of different scales
may be
displayed.}{}{fig:SoftwareDesignOfTheGpstoolPackage-MapsOfDifferentScalesMayBeDisplayed}

The first attempt to draw the maps was the following: Find all maps
that are visible and draw them in the order largest scale to smallest
scale. So if there is a plan of the city Graz and a map of Europe, the
city plan is painted over the map of Europe.

This algorithm scales very badly, as all maps are painted, even if the
user does not see the maps because of another map lying over the first
one.

So an algorithm was developed that searches the smallest map to show,
paint it, and find the rectangles on the screen that are not covered
by this map. For the remaining empty rectangles, the algorithm is
repeated until the screen is filled, or no more maps are
available. This algorithm is implemented and documented in the
\classname{org.dinopolis.gpstool.gui.util.VisibleImage} class.

Maps are only painted, if their scale is not completely different to
the scale that is currently being used. This prevents the painting of
the city plan, when the user wants to see western Europe, as the city
plan would be so small anyway. So if the current scale is 1:200000,
only maps up to (e.g.!) 1:100000 are used, other (more detailed) maps,
are not even considered to be painted! This factor is configurable.

%----------------------------------------------------------------------
%----------------------------------------------------------------------
\subsubsection{Location Marker Layer}
\label{SoftwareDesignOfTheGpstoolPackage-LocationMarkerLayer}

The layer that displays location markers handles different
sources\footnote{interface
\classname{org.dinopolis.gpstool.gui.layer.location.LocationMarkerSource}}
of markers. They may be read from a file or from a relation database
and provide \classname{LocationMarker} objects for a given area
(limited by north, south, west, east latitude/longitude).

Additionally the sources can be asked to apply a given filter, so only
location markers for one or more given categories should be
retrieved. This Filter was designed to be independent of the source,
so in the case of a relation database source it is translated into the
correct SQL statements.


%######################################################################
\subsection{Debug}
\label{SoftwareDesignOfTheGpstoolPackage-Debug}

The \packagename{org.dinopolis} packages use the
\classname{org.dinopolis.util.Debug} package for printing debug
messages. This package is similar to the \packagename{log4j} package
of the apache framework. It allows to define debug messages that are
only printed if the attached debug level is activated.

The debug levels may be activated by using the appropriate API or by
editing the debug properties file. For a detailed description of the
\classname{Debug} class, please see the design document of the debug
utility. 


%% list of acronyms
%\addcontentsline{toc}{chapter}{\listacronymname}
%\printglosstex(acr)


%% ======================================== end document header

%% ======================================== begin document body

%% the document body following the Dino Documentation Rules.
%% Please read the Dino Documentation Guidelines for structuring
%% conventions, etc...

%% ======================================== end document body

%% glossary
%\addcontentsline{toc}{chapter}{\glossaryname}
%\printglosstex(glo)

%% ======================================== begin references

%\bibliographystyle{alpha}
%\addcontentsline{toc}{chapter}{\bibname}
%\bibliography{../../bibliography_entries} 

%% ======================================== end references

%% end of document marker to be able to see if the document is
%% complete when printed

\documentend

\end{document}



