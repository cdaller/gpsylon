%% Template for an article following the Dino Documentation
%% Guidelines.
%%
%% Please read the Dino Documentation Guidelines first before writing
%% any Dino document as it is mandatory to fully make use of the dino.sty
%% document style. This especially applies for the macros in there
%% that define images, references, etc...

\documentclass[a4paper]{article} %% never change options or the
%% document class!!!

\usepackage[dinodraft]{dino}                %% never change this line!!!

%% insert necessary additional packages here - please keep in mind
%% that dino.sty itself already includes the necessary standard
%% packages like graphicx, etc... please read the Dino
%% Documentation Guidelines for details.

%% never change any latex variables like pagestyle, baselineskip,
%% etc... here!!!

\begin{document}
\begin{Form}\end{Form}
%% ======================================== start document header

\title{Software Design of the GPSTool Package}                     %% insert title here
\author{(Christof Dallermassl)\\                         %% insert names of authors here in
  %% Dino Document Style form,
  %% e.g. \cdaller, \hhaub
  $Revision$}                      %% never change this line!!!


\maketitle

\vspace{1cm}

\begin{abstract}

This document describes the rough design of the modules in the
\packagename{org.dinopolis.gpstool} package and the modules of the
\classname{GPSylon} application.

\end{abstract}

\newpage

\tableofcontents

\newpage

%######################################################################
%######################################################################
\section{Architectural Design}
\label{SoftwareDesignOfTheGpstoolPackage-ArchitecturalDesign}

This section describes the modules contained in the
\packagename{org.dinopolis.gpstool} package and explains the structure
of the \classname{GPSylon} application.

%######################################################################
\subsection{GPS Data Sources}
\label{SoftwareDesignOfTheGpstoolPackage-GpsDataSources}

One of the major modules in the \packagename{org.dinopolis.gpstool}
package is the one that reads and interprets data from a gps
device. This module is named
\packagename{org.dinopolis.gpstool.gpsinput}.

The module was designed to be independent of the format of the data
and of the source of the data. An example for different formats of the
data could be NMEA or the proprietary Garmin protocol, the source
could either be the serial port, a file or a network server that
provides any clients with gps data (like
\externalref{gpsd}{http://freshmeat.net/projects/gpsd/} does).

So to be able to get gps information (like position, altitude, speed,
etc.), the source in the form of a \classname{GPSDevice} has to be
chosen and a way to interpret the data coming from the device, in the
form of a \classname{GPSDataProcessor}.

These two classes are connected and from this moment on, gps
information can be obtained. This information is delivered in the form
of events, anyone can register for. The listener can register for all
gps events or just for a specific one. In
\stdref{tab:SoftwareDesignOfTheGpstoolPackage-EventsFiredFromTheGpsdataprocessor}
are the events and its value types listed.


\begin{table}[ht]
  \begin{center}
    \begin{tabular}{|l|l|}
      \hline
      Event Type & Value of Event \\
      \hline
      Location & \classname{GPSPosition}\\
      Heading & \classname{Float}\\
      Speed & \classname{Float}\\
      Number of Satellites & \classname{Integer}\\
      Altitude (in meters)& \classname{Float}\\
      Satellite Info & \classname{SatelliteInfo}\\
      Depth & \classname{Float}\\
      Estimated Pos Error (EPE) & \classname{GPSPositionError}\\
      \hline
    \end{tabular}
  \end{center}
  \caption{Events fired from the GPSDataProcessor}
  \label{tab:SoftwareDesignOfTheGpstoolPackage-EventsFiredFromTheGpsdataprocessor}
\end{table}


A short code snipped shows how to read NMEA data from a serial device:

\begin{lstlisting}{Read gps data from the serial device.}
// create processor for NMEA data:
  GPSDataProcessor gps_data_processor = new GPSNmeaDataProcessor();

// create gps device for serial port:
  Hashtable environment = new Hashtable();
  environment.put(GPSSerialDevice.PORT_NAME_KEY,"/dev/ttyS1");
  environment.put(GPSSerialDevice.PORT_SPEED_KEY,new Integer(4800));
  GPSDevice gps_device = new GPSSerialDevice();
  gps_device.init(environment);

// connect processor with device and open it:
  gps_data_processor.setGPSDevice(gps_device);
  gps_data_processor.open();


// create property change listener for gps events:
PropertyChangeListener listener = new ProperyChangeListener()
{
  public void propertyChange(PropertyChangeEvent event)
  {
    Object value = event.getNewValue();
    String name = event.getPropertyName();
    if(name.equals(GPSDataProcessor.LOCATION))
    { 
      System.out.println("The new location is"
       +(GPSPosition)value.getLatitude() + "/"
       +(GPSPosition)value.getLongitude());
    }
  }
};

// register as listener for location events:
  gps_data_processor.addGPSDataChangeListener(GPSDataProcessor.LOCATION,listener);
\end{lstlisting}

A little example that demonstrates the features of this module is the
java application
\classname{org.dinopolis.gpstool.gpsinput.GPSTool}. It shows how to
read from a file or from the serial interface and how to register for
gps events. As a matter of fact, this application was the beginning of
the whole module.


%######################################################################
\subsection{GPSylon Application}
\label{SoftwareDesignOfTheGpstoolPackage-GpsmapApplication}

GPSylon is the main application of the \packagename{org.dinopolis.gpstool}
package. It is a moving map application that is able to show the
current position on maps that may be downloaded from the internet, a
track of the positions in the past, location markers for points of
interest, etc.

GPSylon uses some parts of the open source
\externalref{openmap}{http://openmap.bbn.com} framework. Although the
openmap framework provides a lot of functionality, some was not reused
but re-implemented to keep the dependencies to the library low.

Nevertheless, GPSylon uses openmap's \classname{MapBean} class as its
central component. A MapBean consists of layers that hold geographic
information to be drawn for a specific area and scale. 

The main class of the GPSylon application is
\classname{org.dinopolis.gpstool.GPSylon}.

%----------------------------------------------------------------------
%----------------------------------------------------------------------
\subsubsection{Resources}
\label{SoftwareDesignOfTheGpstoolPackage-Resources}

GPSylon reads some command line parameters, but most if the
configuration is read from a properties file
(\texttt{Gpsylon.properties}). This file must be in the classpath of
the application and is read via the
\classname{org.dinopolis.util.Resources} class. Any changes of the
configuration are saved into a file into the directory
\texttt{.gpsylon} under the user' home directory. Not all resources can
be edited via the ``Preferences'' dialog, so if you are missing some
screws to turn, try the file itself.

The resources also hold the information for the resource editor
(title, description, type).


%----------------------------------------------------------------------
%----------------------------------------------------------------------
\subsubsection{User Interface}
\label{SoftwareDesignOfTheGpstoolPackage-UserInterface}

The user interface is widely configured in the resource files. The
structure of the menu is completely defined in the resource file and
the actions that are executed by selecting a menu entry are named in
the resource file as well.

Localization can be done by creating a localized version of the
resource file.

%----------------------------------------------------------------------
%----------------------------------------------------------------------
\subsubsection{Projection}
\label{SoftwareDesignOfTheGpstoolPackage-Projection}

This data is projected from the geoid coordinates (latitude,
longitude) to screen coordinates. As the projections provided by
openmap did not work for the maps of
\externalref{mapblast}{http://www.mapblast.com} or
\externalref{expedia}{http://www.expedia.com}, a new projection was
developed. The maths was taken from the
\externalref{gpsdrive}{http://www.gpsdrive.de} project of Fritz
Ganter.

This projection provides the calculation from latitude/longitude to
screen (\methodname{forward} methods) and from screen coordinates to
latitude/longitude (\methodname{inverse} methods).

The class that implements the projection is
\classname{org.dinopolis.gpstool.projection.FlatProjection}. 

For a full understanding of this class it is necessary to read the
documentation of the projections of the openmap framework.

%----------------------------------------------------------------------
%----------------------------------------------------------------------
\subsubsection{Layers}
\label{SoftwareDesignOfTheGpstoolPackage-Layers}

GPSylon organizes its data in layers that are administered by a
\classname{com.bbn.openmap.MapBean}. Whenever the projection changes
(scale or center is changed), the map bean informs all layers about
this change (\methodname{projectionChanged} method). The layers have
to recalculate (project) their data from latitude/longitude to the
screen coordinates and paint them. As the calculation may take its
time, this is usually done in a different task by a
\classname{SwingWorker}. As soon as the calculation is done, the data
is painted on the screen (\methodname{paintComponent} method).

The usage of background tasks also explains the behavior of GPSylon,
that after panning the map, other elements (in other layers) are drawn
slightly later at their correct position.

If one wants to add geographic information (e.g.~position of
friends/cars, etc.) the best solution is to add a new layer that
implements the \methodname{projectionChanged} and the
\methodname{paintComponent} methods. That's all! Using the projection
passed in the \methodname{projectionChanged} method, the conversion of
geographical to screen coordinates is easy. Lengthy calculations
should use a \classname{SwingWorker}, so the user interface is not
blocked. 

In the following, some detailed information about different layers is
given.

%----------------------------------------------------------------------
%----------------------------------------------------------------------
\subsubsection{Map Layer}
\label{SoftwareDesignOfTheGpstoolPackage-Map-Layer}

The map layer is probably the most important layer at the moment. It
displays raster maps that were previously downloaded form expedia or
mapblast and stored locally on the hard disk (directory
\path{<home>/.gpsylon/maps}). The informations about the files is kept
in the file \path{<home>/.gpsylon/maps.txt} (name of file,
latitude/longitude of center of map, scale of map (in mapblast style),
height/width of image). In this file, relative and absolute paths are
accepted for maps.

One principle of the map painting algorithm is that if no maps for a
given scale are available, maps of other scales are used as well and
resized to fit the used scale (see
\stdref{fig:SoftwareDesignOfTheGpstoolPackage-MapsOfDifferentScalesMayBeDisplayed}
for an example).

\image[1.0]{images/screenshot_diff_scales}{Maps of different scales
may be
displayed.}{}{fig:SoftwareDesignOfTheGpstoolPackage-MapsOfDifferentScalesMayBeDisplayed}

The first attempt to draw the maps was the following: Find all maps
that are visible and draw them in the order largest scale to smallest
scale. So if there is a plan of the city Graz and a map of Europe, the
city plan is painted over the map of Europe.

This algorithm scales very badly, as all maps are painted, even if the
user does not see the maps because of another map lying over the first
one.

So an algorithm was developed that searches the smallest map to show,
paint it, and find the rectangles on the screen that are not covered
by this map. For the remaining empty rectangles, the algorithm is
repeated until the screen is filled, or no more maps are
available. This algorithm is implemented and documented in the
\classname{org.dinopolis.gpstool.gui.util.VisibleImage} class.

Maps are only painted, if their scale is not completely different to
the scale that is currently being used. This prevents the painting of
the city plan, when the user wants to see western Europe, as the city
plan would be so small anyway. So if the current scale is 1:200000,
only maps up to (e.g.!) 1:100000 are used, other (more detailed) maps,
are not even considered to be painted! This factor is configurable.

%----------------------------------------------------------------------
%----------------------------------------------------------------------
\subsubsection{Location Marker Layer}
\label{SoftwareDesignOfTheGpstoolPackage-LocationMarkerLayer}

The layer that displays location markers handles different
sources\footnote{interface
\classname{org.dinopolis.gpstool.gui.layer.location.LocationMarkerSource}}
of markers. They may be read from a file or from a relation database
and provide \classname{LocationMarker} objects for a given area
(limited by north, south, west, east latitude/longitude).

Additionally the sources can be asked to apply a given filter, so only
location markers for one or more given categories should be
retrieved. This Filter was designed to be independent of the source,
so in the case of a relation database source it is translated into the
correct SQL statements.


%######################################################################
\subsection{Debug}
\label{SoftwareDesignOfTheGpstoolPackage-Debug}

The \packagename{org.dinopolis} packages use the
\classname{org.dinopolis.util.Debug} package for printing debug
messages. This package is similar to the \packagename{log4j} package
of the apache framework. It allows to define debug messages that are
only printed if the attached debug level is activated.

The debug levels may be activated by using the appropriate API or by
editing the debug properties file. For a detailed description of the
\classname{Debug} class, please see the design document of the debug
utility. 

%######################################################################
%######################################################################
\section{Plugins}
\label{SoftwareDesignOfTheGpstoolPackage-Plugins}

The GPSylon application supports different kinds of plugins. These
plugins extend the functionality of the application. As there is a
clearly and quite simple interface for plugins, the developement is
quite simple and possible without the need to read and understand the
source of the rest of the application. Another advantage of plugins
is, that removing then makes the application smaller and probably
faster. So big, fat computers may install and run all possible
plugins, whereas the old notebook that is used in the car for location
tracking just uses the plugins that are really needed. 

In the following, a short explanation of the different aspects of
plugin developement is given.

%######################################################################
\subsection{Types of Plugins}
\label{SoftwareDesignOfTheGpstoolPackage-TypesOfPlugins}

GPSylon supports different kinds of plugins for different
purposes. Some are very special for a very small task (like saving the
content of the map component to a file (screenshot)), others are very
general and influence the functionaltiy in a wider way (like adding a
layer to the map component, some entries in the menu, and may react on
mouse clicks and keys from the user). In general, all plugin
interfaces (and helper interfaces) are in the package
\packagename{org.dinopolis.gpstool.plugin}.

\begin{itemize}

\item The \interfacename{ReadTrackPlugin} is used to load tracks from
files.

\item The \interfacename{WriteImagePlugin} is used to save a
``screenshot'' from the map component to a file.

\item The \interfacename{MouseModePlugin}: GPSylon supports different
mouse modes that react on mouse activities from the user (click, drag,
etc.). The user may switch a mouse mode on or off (and there is always
only one mouse mode active). An example for a mouse mode is navigation
(zoom in out, pan the map, etc.). As the mouse modes are switched on
or off by the user, they have to provide some information for an entry
in the menu or a toolbar: name, icon, shortcut key, ...

Additonally, a mouse mode may provide a layer that allows the mouse
mode to draw something on the map component.
 
\item The \interfacename{GuiPlugin} is the most powerful plugin. It
  has the possibility to add one or more entries in the menu of the
  application. An example would be a window that provides information
  about memory usage or a tachometer that shows lots of informations
  from the gps device (speed, average, direction, etc.).

Additionally, \interfacename{GuiPlugins} may provide one or more mouse
modes to be able to react on mouse clicks on the map component. 

If a \interfacename{GuiPlugin} needs to draw anything on the map
component, it may provide a \classname{Layer}.
GPSylon uses the \classname{MapBean} class of the openmap library as
the central map component. This class uses \classname{Layer} objects
as layers.  Every layer is informed about changes of the projection
(zoom in/out, move) in the
\methodname{projectionChanged(com.bbn.openmap.proj.Projection)}
method. The layer may paint some map details in the
\methodname{paint(java.awt.Graphics)} or
\methodname{paintComponent(java.awt.Graphics)} methods. Please, see
\stdref{SoftwareDesignOfTheGpstoolPackage-Projection} for details
about projections.

The \methodname{paint()} method should return as fast as possible, so
complicated calculations should not be done in this method but in a
background thread. The class
\classname{org.dinopolis.gpstool.gui.BasicLayer} provides a framework
that calls a calculation method in a background thread
(\classname{SwingWorker}). So the developer does not need to worry
about this!

\item The \classname{MapRetrievalPlugin}: Used by the download mouse
  mode to retrieve raster maps from various sources. The application
  requests a raster map for a given location, size and scale. As
  different map sources may provide only specific scales, the
  \interfacename{MapRetrievalPlugin} has a method that has to return
  the really used scale (may differ from the scale requested). All
  scale values are given in mapblast units (as this was the first
  server supported). E.g.\ 1000 is a very small scale (lots of
  details), whereas 1000000 (one million) shows most of Europe.

\end{itemize}

The interface \interfacename{Plugin} is the base interface for all
plugins that provide general information (like the name and version of
a plugin). This information may be used by a plugin manager or by a
plugin downloader to find the latest version of a plugin.

All plugins are instatiated and then initialized with a
\interfacename{PluginSupport} object that provides interfaces for all
important modules and componentes of GPSylon.

Please see the javadoc documentation of the plugin interfaces and
classes for further details.

%######################################################################
\subsection{Loading of Plugins}
\label{SoftwareDesignOfTheGpstoolPackage-LoadingOfPlugins}

In this section the two mechanisms used to find and load plugins are
explained: 

\begin{itemize}

\item Find the implementation classes of a given interface or base
class. 

\item Load classes from jar files that are not in the classpath.

\end{itemize}

The first problem when using plugins is to find one or more
implementations of a special java interface or base class. The normal
classloader does not provide this functionality, so the help of an
external configuration is needed. Sun uses a file in the
\texttt{META-INF/services} directory (in a jar file or elsewhere in
the classpath) for this purpose. This file is named like the interface
or base class and contains the names of classes that implement this
interface or base class.

As an example, suppose there are two implementations of the interface
\interfacename{org.dinopolis.gpstool.plugin.GuiPlugin}, namely the
classes \classname{foo.Bar} and \classname{bar.Foo}. So the file
\texttt{META-INF/services/org.dinopolis.gpstool.plugin.GuiPlugin}
contains two lines:
\begin{verbatim}
foo.Bar
bar.Foo
\end{verbatim}

Sun uses an internal class (in the package \packagename{sun.misc}) to
find the implementations (called \textit{services}). As the sun class
is internal and may change in the future, its functionality was newly
implemented. \classname{org.dinopolis.util.servicediscovery.ServiceDiscovery}
retrieves the information in \texttt{META-INF/services} from one or
more classloaders and returns the names or faster the instances of the
classes of the given interface of base class.

The \classname{ServiceDiscovery} uses the
\methodname{getResources(String)} of the \classname{ClassLoader} to
find the files in the \texttt{META-INF/services} directory. 

This leads to the second problem of plugins: it cannot be guaranteed
that the classpath includes the plugin (jar). A good plugin
architecture allows to find, load and use the jar files that contain
the plugins in one or more directories. For this purpose, a special
\classname{ClassLoader} was written that loads classes from jar files
located in one or more directories that may be given at runtime.

This class loader is
\classname{org.dinopolis.util.servicediscovery.RepositoryClassLoader}
and in combination with the \classname{ServiceDiscovery} described
above, it provides the wanted functionality: Find and load
implementations of a given class and use jar files in one or more
directories.


%% list of acronyms
%\addcontentsline{toc}{chapter}{\listacronymname}
%\printglosstex(acr)


%% ======================================== end document header

%% ======================================== begin document body

%% the document body following the Dino Documentation Rules.
%% Please read the Dino Documentation Guidelines for structuring
%% conventions, etc...

%% ======================================== end document body

%% glossary
%\addcontentsline{toc}{chapter}{\glossaryname}
%\printglosstex(glo)

%% ======================================== begin references

%\bibliographystyle{alpha}
%\addcontentsline{toc}{chapter}{\bibname}
%\bibliography{../../bibliography_entries} 

%% ======================================== end references

%% end of document marker to be able to see if the document is
%% complete when printed

\documentend

\end{document}



